\documentclass[12pt,a4paper,notitlepage]{extarticle}

\usepackage[utf8]{inputenc}
\usepackage[T1]{fontenc}
\usepackage{amsmath}
\usepackage{amssymb}
\usepackage{lipsum}
\usepackage{url}

\usepackage{noto}
\usepackage[margin=0.5in]{geometry}

\usepackage{listings}
\lstset{
    tabsize=2,
    basicstyle = \ttfamily\small,
    columns=fullflexible
}

\author{Azullia / 0xFC963F18DC21}
\title{\textbf{\huge{Cracking Open Gran Turismo Spec II's Randomizer}}}

\begin{document}
    \maketitle

    \begin{abstract}
        Gran Turismo 4 Spec II is a modification for the NTSC-U release of Gran Turismo 4's
        Online Public Beta. One of its main defining features is a toggleable prize car
        randomizer, which randomizes what cars one receives as prizes for various actions,
        picked based on one's in-game username. We will discuss the inner workings of Gran
        Turismo 4's internal PRNG, and how it is used by Gran Turismo 4 Spec II to generate
        random prize cars.
    \end{abstract}

    \section*{Preface}
        Gran Turismo 4 Spec II is an ISO patch modification for the US soft-release of
        Gran Turismo 4's Online Public Beta created by TheAdmiester. It brings a lot of
        additions and changes to the main game, but one of its main draws / features is
        its prize car randomizer.

        Debuting in 2023 as the initial public Randomizer build, the feature was brought into
        Spec II as a toggleable option. Both versions share the same apparent functionality:
        changing any prize car\footnote{Prize cars are awarded by Gran Turismo 4 typically for
        completing all tests in a License grade with Bronze, Silver or Gold level times,
        winning all races in a single-race event, winning the overall championship of a
        championship events, completing sets of Driving Missions, etc..} into a
        randomly-selected option based on one's in-game username and the circumstances that one
        is winning a car in.

        As with a lot of other randomizer-adjacent modifications for games, of course, there
        will be people who wish to know how it internally works, so as to:

        \begin{itemize}
            \item Find an optimal seed to route through.
            \item Have a really fun or overpowered seed to play through.
            \item Have an intentionally bad seed as a challenge.
        \end{itemize}

        \noindent And many more possible reasons. As a disclaimer however, the method described
        within this paper does not work with the original public Randomizer build from 2023.

    \section*{Pseudorandom Number Generators (PRNGs)}
        With most if not all kinds of ``randomness'' that occurs in computer programs, they are
        never \textit{truly} random, as that is antithetical to how computers work (i.e. by
        following instructions in a sequence). Without external inputs of data\footnote{
        Cloudflare does something cool for their secure random number generators
        involving some lava lamps (
        \url{https://www.cloudflare.com/en-gb/learning/ssl/lava-lamp-encryption/}),
        for example.}, computers cannot ever truly replicate something that is truly and
        irrefutably random\footnote{By \textit{truly} random, this means that any output is
        completely unaffected by any surrounding state or previous and future inputs or
        outputs.}.

        Keeping that in mind, most if not all computer programs use things called \textit{
        Pseudorandom Number Generators}. As their name suggests, they merely ``fake'' or
        ``mimic'' true randomness, and are not actually truly random. Typically, these take
        the form of functions of the form:

        \begin{equation*}
            \text{PRNG}(\text{state}) := (\text{pseudorandom output}, \text{new state})
        \end{equation*}

        And random numbers are generated by repeatedly feeding in the new state\footnote{
        The \textit{state} of a PRNG can be as simple or as complex as it needs to be.} as
        the next inputs to the PRNG algorithm. The pseudorandom output can then be used in
        any way the programmer sees fit in order to generate ``random''-looking values.

        \subsection*{Gran Turismo 4's PRNG Algorithm}
            Gran Turismo 4's main exposed interface for pseudorandom number generation is
            \texttt{MRandom}. This is a class exposed in Adhoc\footnote{Adhoc is a scripting
            language used by Polyphony Digital for certain tasks in Gran Turismo 4 onwards.
            See \url{https://nenkai.github.io/gt-modding-hub/concepts/adhoc/adhoc/} for more 
            information.} that is used in all places where Gran Turismo 4 requires
            random-looking values\footnote{The used car dealer uses \texttt{MRandom}, for
            example, to generate the mileages based on the used car cycle week number and car
            index.}.

        \subsection*{\texttt{MRandom}}
            \texttt{MRandom} tracks a simple state: a single unsigned 32-bit integer (which
            represents a whole number between 0 and 4294967295 inclusive). This state is then
            fed into its PRNG function inside, which outputs and stores a new state, along
            with a value that is the ``random-looking'' value generated alongside. It performs
            this using a small portion of code inspired by the CRC32 algorithm.

            \subsubsection*{CRC32}
                The CRC32 algorithm is a checksumming function. Its job is to take a piece of
                data, and perform some operations to mix groups of bytes in the data together
                in the order the data comes in. Normally, it can be used as a form of file
                integrity check (i.e. checking if a file is undamaged / unmodified), but the
                same property that makes it effective for doing so\footnote{A large cascade
                effect, where a small change in input makes a large change in output.} also
                allows it to be a very simple PRNG algorithm.

            \subsubsection*{So how does \texttt{MRandom} work?}
                Essentially, \texttt{MRandom}'s PRNG function looks like the following:

                \begin{lstlisting}
    def mRandomNext(seed: UInt32): (UInt32, UInt32) = {
      var temp: UInt32   = (17 * seed) + 17
      var result: UInt32 = 0

      for (_ in 0 until 4) {
        val index: UInt8 = result ^ temp
        temp = temp >> 8

        result = (result >> 8) ^ CRC32TABLE(index)
      }

      return (result, (17 * seed) + 17)
    }
                \end{lstlisting}

                It simply operates on 8 bits of the 32-bit integer at a time, calculating a
                lookup index (that wraps around 0 through 255), that points to a value in
                a precalculated CRC32 lookup table of values\footnote{The first table in
                \url{https://web.mit.edu/freebsd/head/sys/libkern/crc32.c}}, both using the
                Exclusive Or\footnote{\url{https://en.wikipedia.org/wiki/Exclusive_or}}
                operation.

                Although, this doesn't explain how we get useable numbers within a range
                from these generated pseudorandom values, since without any extra operations,
                we are stuck with a number anywhere between 0 and 4294967295. This is where the
                next two functions come in.

            \subsection*{\texttt{GetRange} and \texttt{RandomInt32ToFloat}}
                \texttt{GetRange} wraps around the raw PRNG function to allow it to generate
                random whole numbers within a range.

                \begin{lstlisting}
    def getRange(incMin: Int32, excMax: Int32, seed: UInt32): Int32 = {
      val (rval, _)  = mRandomNext(seed)
      val multiplier = randomInt32ToFloat(rval)

      // The decimal of the result of the multiplication is chopped off.
      return ((excMax - incMin) * multiplier + incMin)
    }
                \end{lstlisting}

                \noindent \texttt{RandomInt32ToFloat} translates an unsigned 32-bit integer
                into a decimal value between 0 and 1, where each number between 0 and 4294967295
                is translated evenly into that range (with the caveat that 1 is never returned,
                hence the maximum given in \texttt{GetRange} is exclusive).

    \section*{Spec II Prize Car Randomizer}
        In Gran Turismo 4 Spec II, if the prize car randomizer is enabled, when one would
        normally earn a prize, instead of awarding the prize associated with that instance of
        a prize, it will randomly select a car from a list of all cars available in Spec II.

        How it randomly selects a car is based on the username, the current ``trigger'' that is
        giving the prize car, and what type of trigger it is:

        \begin{itemize}
            \item Username is self-explanatory. It is the up to 30-long string one inputs when
            making a save on a memory card for the first time\footnote{If entering usernames
            there, \textbf{do not} use the HD HUD / UI texture pack, as that incorrectly blanks
            out some of the keys on the improved keyboard.}.
            \item Triggers are a list labels of events that you can win a prize car in, they
            represent things like event hall events (Sunday Cup, etc.), License grade tiers (B
            Bronze, etc.) or Mission Set completions (e.g. One-Lap Magic 30-34).
            \item Types of triggers denote what kind of condition it was activated (an event
            win, mission set completion, license grade completion, etc.).
        \end{itemize}

        \noindent These parts are then put together into one big string, and that is then used
        as the seed whenever one wins a prize (e.g. Username ``Foo'', event Sunday Cup
        (``am\_sunday\_0000'') and type ``win'' is put together to become
        ``Fooam\_sunday\_0000win''). The question is then:

        \begin{center}
            How does it turn that text into a number to feed into \texttt{GetRange}?
        \end{center}

        \subsection*{Fowler-Noll-Vo Hashes}
            A hash function is a way to generate a ``summary'' of a piece of data, similar to
            a checksum. They also ideally exhibit the same property as a checksum, where a
            small change in input data creates a massive change in the final hash output.

            Gran Turismo 4 Spec II uses a variant of the Fowler-Noll-Vo algorithm, a very simple
            to implement hashing algorithm, especially for text / strings. Its implementation
            in Spec II is specifically the FNV-1a variant, which looks like:

            \begin{lstlisting}
    def fnv1a(str: String): UInt32 = {
        // These starting values can usually be substituted for other
        // suitable values in other implementations of FNV-1a.
        val prime: UInt32  = 16777619
        var result: UInt32 = 0x811C9DC5

        for (chr in str) {
            result ^= chr.toUInt8
            result *= prime
        }

        return result
    }
            \end{lstlisting}

            \noindent So the example string such as ``Fooam\_sunday\_0000win'' is then passed
            into \texttt{fnv1a} (which in this case becomes 3244945079), and then the result is
            passed into \texttt{GetRange} as the seed.

        \subsection*{Final Algorithm}
            Putting it all together, we have the following algorithm:

            \begin{lstlisting}
    extern val allCars: List[Car]

    def randomPrize(username: String, trig: String, trigType: String): Car = {
        val seed  = fnv1a(username + trig + trigType)
        val index = getRange(0, allCars.length, seed)

        return allCars(index)
    }
            \end{lstlisting}

            \noindent A very simple algorithm overall, but very effective at its objective of
            delivering the player random prize cars based on username\footnote{For cars with
            multiple possible colours, the seed fed to \texttt{GetRange} is based on the PS2's
            system clock.}, while being able to choose the same car again should a player
            decide to (e.g.) repeat an event they've won a prize car before.

    \section*{Appendices}
        \subsection*{Spec II Adhoc Sources}
            They can be found on TheAdmiester's GitHub here:
            \url{https://github.com/TheAdmiester/OpenAdhoc-GT4SpecII}.

            Please be warned that some amount of programming experience will be needed to
            understand the code, and a primer can be found on Nenkai's GT Modding hub at
            \url{https://nenkai.github.io/gt-modding-hub/concepts/adhoc/adhoc/}

        \subsection*{Analyzer Tool Sources}
            They can be found at \url{https://github.com/MF42-DZH/GT4S2RC}.

        \subsection*{Seed Bruteforcing}
            With the possibility to now check the cars found in a seed for Spec II's
            Randomizer, it becomes possible to look though seeds (many, many seeds) in order
            to try and find a seed optimised for some purpose.

            TeaKanji was and is a major help in this process, providing both approximate
            ``viability'' values for all cars in Spec II, and a list of cars that are essential
            for a 100\% run using just prize cars supplied by the randomizer. We've both gone
            through likely millions of usernames, checking their prizes to see if we can either:

            \begin{itemize}
                \item Find a seed with consistently overpowered car drops.
                \item Find a seed where it is possible to 100\% the game in with just prize cars.
            \end{itemize}

            The first goal was achieved relatively comfortably by trying to find seeds with
            a high average car viability, some examples include:

            \begin{itemize}
                \item \texttt{GatewomanBoundaries} (with its \textbf{five} 787Bs!)
                \item \texttt{ZaniestDisconcert} (with its \textbf{B License Bronze} Formula
                Gran Turismo!)
                \item \textbf{[REDACTED AT THE REQUEST\footnote{(at gunpoint)} OF CERO, PEGI,
                THE ACB, AND THE ESRB]}
                \item \texttt{std::reinterpret\_cast<float>} (with its \textbf{five} BMW V12
                LMRs!)
            \end{itemize}

            The next goal, which we have still not met, is to find a username that is able to
            100\% the game without any bought cars. The closest we have gotten is five cars
            needing to be bought for a candidate seed, one such example being
            ``ItDoESn{\textquotesingle}tHAVEtoBEthiSWaY'' (case-sensitive).

            Feel free to use the bruteforcers included in the repository (or write your own) to
            try and attempt to find that mythical 100\%-able seed!

        \subsection*{Acknowledgements}
            \small{
                \begin{itemize}
                    \item \textbf{Nenkai}: Disassembly Help and GTHD Debug Executables Archive
                    \item \textbf{TeaKanji}: Car Viability Info, 100\% Necessities Info,
                    Bruteforcing Script Help
                    \item \textbf{beardypig, astrelsky}: The \texttt{ghidra-emotionengine} scripts
                    for Ghidra
                    \item \textbf{mrxbas, Yellowbird}: General Support
                    \item \textbf{TheAdmiester}: Creating Gran Turismo 4 Spec II
                \end{itemize}
            }
\end{document}
